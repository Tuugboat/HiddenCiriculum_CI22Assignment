\documentclass{article}

% Math packages
\usepackage{amsmath}
\usepackage{amssymb}

% R Table packages
\usepackage{booktabs}
\usepackage{float}
\usepackage{colortbl}
\usepackage{xcolor}

% Page layout libraries
\usepackage{a4wide}
\usepackage{setspace}
\usepackage{geometry}
\usepackage{parskip}
\usepackage{fancyhdr}

% Setting up the heading style preferred here
\pagestyle{fancy}
\fancyhead[L]{\thepage}
\fancyhead[C]{}
\fancyhead[R]{\textrm{Robert Petit}}
\fancyfoot[L, C, R]{}


% this package helps us with including images. Setting the graphics path makes it easier to refer to things in the \includegraphics command.
\usepackage{graphicx}
\graphicspath{ {../figures/} }

% Titling
\title{Incarceration Analysis}
\author{Robert Petit}
\date{February, 2022}

\begin{document}
\maketitle
Some ideas for writing beyond these basic figures and tables:
\begin{itemize}
\item  First and foremost, it may be a good idea to go month by month and get the starting date of arrests. 
\item There is also the option to count the total number of incarcerations. This may be redundant to arrests
\item Additionally, if there is the option in NLSY97, it might be neat to look at conviction rates based on race by constructing a conviction likelihood by arrests -> incarcerations
\end{itemize}

\begin{figure}[H]
    \begin{center}
        \includegraphics[width=.85\textwidth]{MonthsIncarcerated_by_racegender}
    \end{center}
    \caption{Happy Cappy}
    \label{fig:graph}
\end{figure}


\input{../tables/MonthsIncarcerated_by_racegender.tex}

\input{../tables/regress_MonthsIncarcerated_by_racegender.tex}

\end{document}